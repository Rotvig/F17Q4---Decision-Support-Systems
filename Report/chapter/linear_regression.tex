\chapter{Linear Regression}
\label{chp:linreg}

Linear regression is an approach in supervised learning. It is used for predicting a quantitative response and it is widely used for statistical learning methods. It may seem that it is a bit simple compared to some of the more modern statistical learning approaches, but it serves as a good base as newer approaches can be seen as an generalization or extensions of the linear regression. 
In this chapter the key ideas in linear regression model and it will describe the least squares approach
which is the most commonly used to fit this model

\section{Lab 3.6.2 - Simple Linear Regression}

Simple linear regression is used for predicting a quantitative response, that is often called Y, based on a single predictor, which is often called X.
The relationship between X and Y can be expressed as:

\begin{center}
	$Y \approx \beta_0 + \beta_1X$
\end{center}

This means that Y is approximately modeled as X or in other word Y regressing on X. 
$\beta_0$ and $\beta_1$ are two unknown constant that represent the intercept and slope in the linear model. They are also known as model coefficients or parameters. 
To be able to predict, the training data is used to produce
an estimation of $\hat{\beta_0}$ and $\hat{\beta_1}$ for the model coefficients. 
The mathematically expression for the computation is:

\begin{center}
	$\hat{\beta_0} = \hat{\beta_0} + \hat{\beta_1}x$
\end{center}

$\hat{y}$ is the prediction of Y based on X = x. The \textit{hat} symbol is used to estimate value for an unknown parameter or coefficient or to predict a value of the response.

\newline
In Lab 3.6.2 the data set Boston is given. It consists of records about 506 neighborhoods around Boston. The task is to predict \emph{medv} (median house  value) via the predictor \emph{lstat} (percent of households with low socioeconomic status).
To perform the linear regression the library sklearn is used in python. 

\lstset{}
\begin{lstlisting}[caption={Python Linear Regression function}, label=lst:lin_reg, mathescape=true]
regr = linear_model.LinearRegression()
regr.fit(predXData, predYData)
\end{lstlisting}

In listing \ref{lst:lin_reg} it is shown how a linear regression object is created, and how it is used to fit the data. ‘predXData’ is the \emph{lstat} data, and ‘predYData’ is the \emph{medv} data.
The result are plotted in a graph seen in figure \ref{fig:lin_reg_plot}.

\myFigure{lin_reg_plot.PNG}{Linear regression plot of \emph{medv} and \emph{lstat}}{fig:lin_reg_plot}{0.6}
 

\section{Lab 3.6.3}
