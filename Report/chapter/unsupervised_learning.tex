\chapter{Unsupervised Learning}
\label{chp:unsuplea}
In unsupervised learning there is no response variable Y. Ther is only a set of features $X_1, X_2,..., X_p$ measured on n observations. So the goal is to find interesting things about the measurements on the features. In this section about the unsupervised learning technique called \emph{clustering} which is used to discover unknown subgroups in data.
Compared to supervised learning unsupervised learning comes with some challenges.the tasks tends to be more subjective and there is no simple goal such as prediction of a response as there is in supervised learning. Furthermore there is no universally accepted mechanism for performing cross validation and how to validate the results on a data set due to there is no Y to check results up against with.
But it is an important tool to predict subgroups when the number of them are unknown.  
\chapter{K-means clustering}
\label{chp:clus}

\subsection{Lab 10.5.1}

\chapter{Hierarchical clustering}

\subsection{Lab 10.5.2}