\chapter{Bootstrap}
\label{chp:boots}

The bootstrap is an extremely powerful statistical tool that can be used to quantify the uncertainty associated with a given estimator or statistical learning method.

It can be used to estimate the standard errors of the coefficients from a linear regression fit. This is just a simple example. However, the power of the bootstrap lies in the fact that it easily can be applied to a wide range of statistical learnings methods.

Bootstrap relies on random sampling with replacement. Replacement means that the same observations can occur more than once or maybe not at all in the bootstrap data set.

The reason for doing this is if there is a case where there aren't enough samples. Then bootstrap can be used to generate simulated data from the original data set.
 

\myFigure{bootstrap_example.PNG}{A graphical example of the bootstrap approach on a small sample containing n = 3 observations}{fig:bootstrap}{0.8}

Figure \ref{fig:bootstrap} illustrates an example of the bootstrap approach on a small sample containing n = 3 observations. Each bootstrap contains n observations. These observations are sampled with replacement from the original data set. Each bootstrap data set is used to obtain an estimate of $\alpha$.
\section{Lab 5.3.4}