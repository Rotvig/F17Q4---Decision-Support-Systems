\chapter{Unsupervised Learning}
\label{chp:unsuplea}
In unsupervised learning there is no response variable Y. There is only a set of features $X_1, X_2,..., X_p$ measured on n observations. So the goal is to find the interesting parts of the measurements on the features.
Supervised learning compared to unsupervised learning comes with some challenges. The tasks tends to be more subjective and there is no simple goal such as prediction of a response as there is in supervised learning. Furthermore there is no universally accepted mechanism for performing cross validation and how to validate the results on a data set, due to the fact that there is no Y to check results up against with.
But it is an important tool to predict subgroups when the number of subgroups are unknown.
In this section the focus will be on the  technique called \emph{clustering} which is used to discover \emph{subgroups} or \emph{clusters} in a data set. 
This is done by dividing the data set into clusters so that the observations in each of the clusters are similar to each other, and the observations belonging to other clusters are different compared to the observation other clusters.
The problem is to determine if observations are similar or different to each other.
 
\section{K-Means Clustering}
\label{chp:clus}
K-means clustering is a technique to divide some observations in a data set into a pre-specified \emph{k} number of clusters. When \emph{k} is specified it will give each of the observations to one of the clusters within 1 - \emph{k}. To understand this it is easier to look at a example. In
Figure \ref{fig:K-meanex} a data set is assigned to three clusters. This means that \emph{k} is equal to 3. The data set consist of 50 observation. Each cluster is marked with a color and the cross is the center points of a cluster. This point is also called \emph{centroids}.

\myFigure{K-mean.PNG}{Example of the K-mean algorithm where K is 3}{fig:K-meanex}{0.7}

K-Means Clustering needs to satisfy two properties. namely
\begin{itemize}
	\item Each observation belongs to one of the clusters
	\item Clusters cannot overlap, meaning that no observation can belong to more than one cluster.
\end{itemize} 

To get a good clustering the \emph{within-cluster variation} needs to be as small as possible. So the algorithm will divide the observation into K clusters, such that the total sum of all \emph{within-cluster variation} is as small as possible. The \emph{within-cluster variation} is commonly calculated with the \emph{squared Euclidean distance}. To obtain the correct result all Euclidean distances are divided by the number of observations in the given cluster. Now it will try to minimize the total \emph{within-cluster variation} summed over all clusters.
The K-means Clustering algorithm can then be written as.

\begin{algorithm}
	\caption{K-Means Clustering}
	\label{algo:KMeansClustering}
	\begin{algorithmic}[1]
		\State A random number from 1 to K is chosen for each observations. This number will act as a initial cluster assignment for every given observation.
		
		\State 
		Iterate until the observations stops changing between clusters.
		\begin{enumerate}[label=(\alph*)]
			\item The centroids are calculated for each of the K clusters and these centroids acts as the vector for the means for each observations in the k'th cluster
			
			\item Each observation the Euclidean distance to each centroid are calculated and the observation are assign to the closest one.
		\end{enumerate}
	\end{algorithmic}
\end{algorithm}

\subsection{Lab 10.5.1}
First a random data set consisting of 50 observations is generated. After this, the \emph{KMeans} from the sklearn library is used to perform the K-mean clustering.
After the algorithm is used, the labels and the centroids are saved. The code is showed in listing \ref{lst:kmeanscode} below with number of clusters are set to k = 2.
\begin{lstlisting}[language=Python, label=lst:kmeanscode, caption=The KMean function and the code which saves the labels and the centroids]
k = 2
kmeans = KMeans(n_clusters=k, init="k-means++",n_init= 20).fit(zip(X,Y))
Labels = kmeans.labels_
centroids = kmeans.cluster_centers_
\end{lstlisting}

The Kmeans also takes two other parameters. namely the \emph{init} and the \emph{n\_init}. Init states which approach that is used, and in this case the \emph{k-means++} approach are used. The n\_init is how many time the approach should be performed. Here it is set to 20.
Labels contains all of the observations and centroids contains the center of the clusters.

Now the task is to plot the labels and the centroids so the two clusters can be visualized. 
This is done with this code shown in listing \ref{lst:kmeansplot}.

\begin{lstlisting}[language=Python, label=lst:kmeansplot, caption=The code for plotting the two clusters]
lines = plt.plot(centroids[:,0], centroids[:,1], 'kx')
plt.setp(lines, ms=15.0)
plt.setp(lines, mew=3.0)

for i in range(k):
dsx = X[np.where(Labels==i)]
dsy = Y[np.where(Labels==i)]
plt.plot(dsx, dsy, 'o')

plt.show()
\end{lstlisting}

First the two centroids are plotted. this is done from line 1 to 3. The \emph{'kx'} is written so they are labeled with a cross.

For plotting the observations a for loop is used. It runs through the two possible clusters. then it takes the X and Y values for the observations assigned to the current cluster. This happens from line 5 - 7 and from line 8-10 the observations are plotted and shown. the \emph{'o'} is added so they are plotted as a dot. 
The result of this code can be seen in figure \ref{fig:kmeansResult}.

\myFigure{KmeansResult.PNG}{The plotted result of lab 10.5.1}{fig:kmeansResult}{0.6}

The centroids are marked with crosses and each observation are either a blue or a green dot depending on which cluster it is assigned too.


\section{Hierarchical clustering}
Hierarchical clustering does not require a pre-specified \emph{k} value as K-means clustering has to.
Furthermore Hierarchical clustering produce a \emph{dendrogram} which shows the observations as a tree structure.
The most common type is called the \emph{agglomerative} clustering, it is also known as the \emph{bottom-up} approach.

\subsection{Dendrograms}
The easiest way of explaining a dendrogram is to show one. So figure \ref{fig:dendrogram} shows a dengrogram which is constructed of a data set with 50 observations.
The labels are shown in different colors. But if these labels where not available and the data was observed without them then all labels will be in the same color. 
One of the leafs is equal to one of the 50 observations. 
Depending on the data set and the observations the dendrogram can help to determine the number of clusters that fits the data best. In this example, the number of cluster are two. Line no. one shows the cut in the dendrogram to get the no. of cluster to be two. Line no. two shows a different cut, and if this was best for the data set the number of clusters would be three.
Observations that fuse close to the top of dendrogram is often more different from each other compared to the observations in the bottom of the dendrogram. 

\myFigure{Dendrogram.PNG}{Dendrogram with 50 observations}{fig:dendrogram}{1}

\FloatBarrier
To show how this works, a simple data set containing 7 observations are shown in figure \ref{fig:dendrogramdata} to the right.
It can be seen that the observations 5 and 7 are close to each other, and observations 1 and 6 are also close to each other. These are the bottom of the dendrogram, as it can been seen in the left side of figure \ref{fig:dendrogramdata}. Now the dendrogram is expanded, and by looking at the next observation that are closest to 5 and 7. Clearly 3 is the closest one to them, so this is the next leaf in the dendrogram. The same is done for observation 1 and 6 and at the end the dendrogram will look like figure \ref{fig:dendrogramdata} to the left.

\myFigure{DendrogramData.PNG}{Illustration of a dendrogram and its data set}{fig:dendrogramdata}{1}

\FloatBarrier
\subsection{The Hierarchical Clustering Algorithm}
to produce the dendrogram the Hierarchical clustering algorithm is used. First step is to find the Euclidean distance between the pair of observation and find the dissimilarities.
Then it runs iteratively trough all the observations. It starts at the bottom and finds the two observations who act as their own cluster, that are closest and fuses those two, so that there now are $n-1$ clusters. This continues, so the next two clusters that are most similar are fused so the number of cluster are now $n-2$. This is done until all observations belongs to one big cluster.

The Hierarchical Clustering algorithm is shown below.

\begin{algorithm}
	\caption{Hierarchical Clustering}
	\label{algo:HierarchicalClustering}
	\begin{algorithmic}[1]
 		\State First the number n of observations and the measure of the $(\dfrac{n}{2}) =n(n-1)/2 $ pairwise dissimilarities. Each of the observations will treat this as its cluster.
 		
 		\State for \emph{i} $= n,n-1,...,2:$
 		\begin{enumerate}[label=(\alph*)]
 			\item All pairwise inter-cluster dissimilarities matching the $i$ cluster are examined and then the pairs that are most similar identified and fused into two clusters. This indicates at which height to place the fusion in the dendrogram.
 			\item Then the new pairwise inter-cluster matching $\emph{i}-1$ remaining cluster are computed.
 		\end{enumerate}
 	\end{algorithmic}
 \end{algorithm}

But one issue has bot been addressed. How does the algorithm determine that cluster 5 and 7 should be fused with cluster 3? The algorithm needs to be cable of define the dissimilarities between two cluster if one or both contains multiple observations. This is done with a extension by using \emph{linkage} which defines dissimilarities between two groups of observations. The most common types of linkage is \emph{complete}, \emph{average}, \emph{single} and
\emph{centroid}.

These are explained in table \ref{table:linkage}

\begin{table}
	\begin{center}
		\begin{tabular}{ | l | p{12cm} |}
			\hline
			Linkage & Description \\ \hline
			Complete & Maximal intercluster dissimilarity. Record the largest of these dissimilarities in cluster A and B \\ \hline
			Single & Minimal intercluster dissimilarity. Record the smallest of these dissimilarities in cluster A and B \\ \hline
			Average & Mean intercluster dissimilarity. record the average of pairwise dissimilarities in cluster A and B\\
			\hline
			Centroid & Dissimiliarity between the centroid for cluster A and B and record the average of these dissimilarities
			\\
			\hline
		\end{tabular}
	\end{center}
	\caption{Different linkage methods}
	\label{table:linkage}
\end{table}


Complete and average linkage are generally the most preferred linkage as they produce more balanced dendrograms. 

Sometimes you need to scale your data set to obtain standard deviation. Then there arise new problems, because which dissimilarity measure should be used and which linkage should be used. This is a difficult problem and there is no standard answer.

\subsection{Lab 10.5.2}
A data set containing 50 observations are randomly generated and assigned in X and Y coordinates. So it is the same data set structure from lab 10.5.1.
The hierarchical clustering dendrogram will be produced using complete, single and average linkage clustering with Euclidean distance.
To perform the hierarchical clustering the library scipy is imported. This provides the functions for creating dendrograms with the linkage that is needed. 
The code for creating the three types of linkage are shown in listing
\ref{lst:Linkage}

\begin{lstlisting}[language=Python, label=lst:Linkage, caption=The code that apllies the three different linkage on the data set]
complete = linkage(zip(X,Y), method='complete', metric='euclidean')
single = linkage(zip(X,Y), method='single', metric='euclidean') 
average = linkage(zip(X,Y), method='average', metric='euclidean')
\end{lstlisting}

As it can been seen, the linkage functions look a lot like each other. The only difference is the method. The two other parameters are the data set and the one that specifies which metric is should use. In this lab the linkage uses Euclidean distance. 
Now it is time for creating the dendrograms. To create them the function \emph{dendrogram} from the scipy library is used. The only parameter that is needed to build them is the three linkage objects that was made in listing \ref{lst:Linkage}

When plotting the dendrograms The only thing that differs is which linkage object that are send as a parameter to the dendrograms function.
The result of each of the three different dendrogram can be seen in figures \ref{fig:dendrogramsCompAvg} and \ref{fig:dendrogramsingle}


\mySubFigure{DendrogramComplete.PNG}{DendrogramAverage.PNG}{Dendrograms for linkage methods complete and Average}{Complete}{Average}{fig:dendrogramsCompAvg}{fig:denComplete}{fig:denAverage}

%\myFigure{DendrogramComplete.PNG}{The dendrogram for the linkage method complete}{fig:dendrogramcomplete}{0.5}

\myFigure{DendrogramSingle.PNG}{Dendrogram for the linkage method single}{fig:dendrogramsingle}{0.5}

%\myFigure{DendrogramAverage.PNG}{The dendrogram for the linkage method average}{fig:dendrogramaverage}{0.5}

As expected the dendrograms are different from each other.
Looking at figures \ref{fig:denComplete} and \ref{fig:denAverage} Average and Complete looks more identical and correct compared to figure \ref{fig:dendrogramsingle} where the method single are used. Here it can be seen that one observation belongs to its own cluster. This can mean that maybe two clusters are not the best solution, and the data set should maybe be divided into more clusters. In this case four cluster would be better.