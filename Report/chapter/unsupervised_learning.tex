\chapter{Unsupervised Learning}
\label{chp:unsuplea}
In unsupervised learning there is no response variable Y. There is only a set of features $X_1, X_2,..., X_p$ measured on n observations. 
Supervised learning compared to unsupervised learning comes with some challenges. The tasks tends to be more subjective and there is no simple goal such as prediction of a response as there is in supervised learning. Furthermore there is no accepted method for determining the quality of the result, due to the fact that there is no Y to check results up against.
In this section the focus will be on the  technique called \emph{clustering} which is used to discover subgroups or clusters in a data set. 
This is done by dividing the data set into clusters so that the observations in each of the clusters are similar to each other, and the observations belonging to clusters are different compared to the observations in other clusters.
The problem is to determine if observations are similar or different to each other.
 
\section{K-Means Clustering}
\label{chp:clus}
K-means clustering is a technique to divide observations in a data set into a prespecified k number of clusters. Each observation is assigned to one of the k clusters. In figure \ref{fig:K-meanex} a data set is assigned to three clusters. This means that \emph{k} is equal to 3. The data set consists of 50 observations. Each cluster is marked with a color and the cross is the center points of a cluster. This point is also called a \emph{centroid}.

\myFigure{K-mean.PNG}{Example of the K-means algorithm where k is 3}{fig:K-meanex}{0.55}

K-means clustering needs to satisfy two properties. 
\begin{itemize}
	\item Each observation belongs to one of the clusters
	\item Clusters cannot overlap, which means that no observation can belong to more than one cluster
\end{itemize} 

To get a good clustering the within-cluster variation needs to be as small as possible. So the algorithm tries to minimize this. The within-cluster variation is commonly calculated with the sum of squared pairwise Euclidean distance between observations and the centroid they belong to, divided by the number of observations belonging to that centroid. 
The K-means clustering algorithm can then be written as in algorithm \ref{algo:KMeansClustering}. Other initialization algorithms than the one described in line 1 exists such as K-means++ \citep{kplusplus}.

\begin{algorithm}
	\caption{K-Means Clustering}
	\label{algo:KMeansClustering}
	\begin{algorithmic}[1]
		\State A random number from 1 to k is chosen for each observation. This number will act as an initial cluster assignment for every given observation.
		
		\State 
		Iterate until the observations stop changing between clusters.
		\begin{enumerate}[label=(\alph*)]
			\item The centroids are calculated for each of the k clusters and these centroids acts as the vector for the means for each observation in the k'th cluster.
			
			\item For each observation the Euclidean distance to each centroid is calculated and the observation is assigned to the closest one.
		\end{enumerate}
	\end{algorithmic}
\end{algorithm}

\subsection{Lab 10.5.1 - K-Means Clustering}
First, 50 random two dimensional data points with a normal distribution is generated. Then the mean is shiftet for 25 of the observations such that two clusters should exists. After this, the K-means from the sklearn library is used to perform the K-means clustering.
After the algorithm is used, the labels and the centroids are saved. The code is shown in listing \ref{lst:kmeanscode} with number of clusters set to k = 2.
\begin{lstlisting}[language=Python, label=lst:kmeanscode, caption=The KMeans function]
k = 2
kmeans = KMeans(n_clusters=k, init="k-means++",n_init= 20).fit(zip(X,Y))
Labels = kmeans.labels_
centroids = kmeans.cluster_centers_
\end{lstlisting}

The K-means also takes two other parameters. Namely the \emph{init} and the \emph{n\_init}. Init states which approach is used, and in this case the \emph{K-means++} approach is used. The n\_init is how many times the approach should be performed, this is set to 20. The algorithm then chooses the best initialization of the 20 that was run.

The clusters are visualized with the code in listing \ref{lst:kmeansplot}.

\begin{lstlisting}[language=Python, label=lst:kmeansplot, caption=The code for plotting the two clusters]
lines = plt.plot(centroids[:,0], centroids[:,1], 'kx')
plt.setp(lines, ms=15.0)
plt.setp(lines, mew=3.0)

for i in range(k):
	dsx = X[np.where(Labels==i)]
	dsy = Y[np.where(Labels==i)]
	plt.plot(dsx, dsy, 'o')

plt.show()
\end{lstlisting}

Initially the two centroids are plotted. This is done from line 1 to 3. The \emph{kx} is written so they are labeled with a cross. For plotting the observations a for loop is used. It runs through the two possible clusters. The \emph{o} is added so the observations are plotted as a dot. 
The result of this code can be seen in figure \ref{fig:kmeansResult}.

\myFigure{KmeansResult.PNG}{The plotted result of lab 10.5.1}{fig:kmeansResult}{0.6}

The centroids are marked with crosses and each observation is either blue or green depending on which cluster it is assigned too. The observations are clusters as expected but only because k was set correctly. If the true number of clusters were not known, then choosing k would have been a more difficult task, this is often the case when clustering is needed.


\section{Hierarchical Clustering}
Hierarchical clustering does not require a prespecified k value as K-means clustering does.
Furthermore hierarchical clustering produces a dendrogram which shows the observations as a tree structure.
The most common type is called the agglomerative clustering, it is also known as the bottom-up approach.

\subsection{Dendrograms}
Figure \ref{fig:dendrogram} shows a dengrogram which is constructed of a data set with 50 observations.
One leaf is equal to one of the 50 observations. Observations that fuses into branches indicates similarity in the observations. The observations that fuse lowest in the dendrogram are more similar.

Depending on the data set and the observations, the dendrogram can help determine the number of clusters that fits the data best. Line number one shows where to cut the dendrogram to divide the observations into two clusters. Line number two shows the cut dividing the observations into three clusters. Looking at the dendogram, two clusters would yield the best result.


\myFigure{Dendrogram.PNG}{Dendrogram with 50 observations}{fig:dendrogram}{1}

\FloatBarrier
To show how this works, a simple data set containing 7 observations is shown in figure \ref{fig:dendrogramdata} to the right.
It can be seen that the observations 5 and 7 are close to each other, and observations 1 and 6 are also close to each other. These are at the bottom of the dendrogram, as it can been seen in the left side of figure \ref{fig:dendrogramdata}. Now the dendrogram is expanded, and by looking at the next observation that are closest to 5 and 7. Clearly 3 is the closest one, so this is the next leaf in the dendrogram. The same is done for observations 1 and 6 and at the end the dendrogram will look like figure \ref{fig:dendrogramdata}.

\myFigure{DendrogramData.PNG}{Illustration of a dendrogram and its data set}{fig:dendrogramdata}{1}

\FloatBarrier
\subsection{The Hierarchical Clustering Algorithm}
To produce the dendrogram the hierarchical clustering algorithm is used. First step is to find the Euclidean distance between the pair of observations and find the dissimilarities.
Then it runs iteratively trough all the observations. It starts at the bottom and finds the two observations acting as their own cluster, that are closest and fuses those two, so that there now are $n-1$ clusters. This continues, so the next two clusters that are most similar are fused, making the number of clusters $n-2$. This is done until all observations belong to one big cluster.

The hierarchical clustering algorithm is shown in algorithm \ref{algo:HierarchicalClustering}.

\begin{algorithm}
	\caption{Hierarchical Clustering}
	\label{algo:HierarchicalClustering}
	\begin{algorithmic}[1]
 		\State Given n observations, measure the pairwise dissimilarities for all  $(\dfrac{n}{2}) =n(n-1)/2 $. Treat each observation as its own cluster.
 		
 		\State for \emph{i} $= n,n-1,...,2:$
 		\begin{enumerate}[label=(\alph*)]
 			\item Examine the pairwise dissimilarities among the i clusters and identity the pair which is most similar and fuse them. The pairwise dissimilarity describes at which height the fuse should be placed in the dendogram.
 			\item Compute the new pairwise inter-cluster dissimilarity for the remaining $\emph{i}-1$ clusters.
 		\end{enumerate}
 	\end{algorithmic}
 \end{algorithm}

One issue has not been addressed. The algorithm needs to be capable of defining the dissimilarities between two clusters if one or both contains multiple observations. This is done with an extension called linkage which defines dissimilarities between two groups of observations. The most common types of linkage are complete, average, single and
centroid. These are explained in table \ref{table:linkage}

\begin{table}
	\begin{center}
		\begin{tabular}{ | l | p{12cm} |}
			\hline
			Linkage & Description \\ \hline
			Complete & Maximal intercluster dissimilarity. Record the largest of the pairwise dissimilarities between observations in cluster A and B \\ \hline
			Single & Minimal intercluster dissimilarity. Record the smallest of the pairwise dissimilarities between observation in cluster A and B \\ \hline
			Average & Mean intercluster dissimilarity. Record the average of the pairwise dissimilarities between the observations in cluster A and B\\
			\hline
			Centroid & Dissimilarity between the centroids/means for the clusters A and B
			\\
			\hline
		\end{tabular}
	\end{center}
	\caption{Different linkage methods}
	\label{table:linkage}
\end{table}


Within statistics complete and average linkage are generally the most preferred linkage as they produce more balanced dendrograms \citep[pp. 394-395]{ISLR}. 

%Sometimes you need to scale your data set to obtain standard deviation. Then new problems arises, because of which dissimilarity measure should be used and which linkage should be used. This is a difficult problem and there is no standard answer.

\subsection{Lab 10.5.2 - Hierarchical Clustering}
A random data set is generated the same way as in lab 10.5.1.
The hierarchical clustering dendrogram will be produced using both complete, single and average linkage with Euclidean distance for measuring dissimilarity.
To perform the hierarchical clustering the library scipy is imported.
The code for creating the three types of linkage are shown in listing \ref{lst:Linkage}.

\begin{lstlisting}[language=Python, label=lst:Linkage, caption=The code that applies the three different linkage on the data set]
complete = linkage(X, method='complete', metric='euclidean')
single = linkage(X, method='single', metric='euclidean') 
average = linkage(X, method='average', metric='euclidean')
\end{lstlisting}

Three parameters are passed: the data set, the method describing the chosen linkage algorithm and metric. The result of each of the three different dendrograms can be seen in figures \ref{fig:dendrogramsCompAvg} and \ref{fig:dendrogramsingle}.
%To create the dendrograms the function \emph{dendrogram} from the scipy library is used. The only parameter that is needed to build the dendrograms is the three linkage objects that was made in listing \ref{lst:Linkage}.

%When plotting the dendrograms the only difference is which linkage object is parsed as a parameter.



\mySubFigure{DendrogramComplete.PNG}{DendrogramAverage.PNG}{Dendrograms for linkage methods complete and Average}{Complete}{Average}{fig:dendrogramsCompAvg}{fig:denComplete}{fig:denAverage}


\myFigure{DendrogramSingle.PNG}{Dendrogram for the linkage method single}{fig:dendrogramsingle}{0.7}

As expected the dendrograms are different from each other.
Looking at figures \ref{fig:denComplete} and \ref{fig:denAverage} average and complete looks more balanced compared to figure \ref{fig:dendrogramsingle} where the method single is used. Both average and complete also placed the 25 observations correctly in each their cluster. Single instead placed one observations in its own and 49 observations in the other. By looking at the dendogram it can also be seen that the three observations on the left should probably be placed in two different clusters, making a total of four clusters. Using four clusters with the single linkage method also gave better result. Now only four observations were placed wrongly in a cluster.